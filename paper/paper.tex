\documentclass[9pt,twocolumn,twoside,]{pnas-new}

% Use the lineno option to display guide line numbers if required.
% Note that the use of elements such as single-column equations
% may affect the guide line number alignment.


\usepackage[T1]{fontenc}
\usepackage[utf8]{inputenc}

% tightlist command for lists without linebreak
\providecommand{\tightlist}{%
  \setlength{\itemsep}{0pt}\setlength{\parskip}{0pt}}


% Pandoc citation processing
\newlength{\cslhangindent}
\setlength{\cslhangindent}{1.5em}
\newlength{\csllabelwidth}
\setlength{\csllabelwidth}{3em}
\newlength{\cslentryspacingunit} % times entry-spacing
\setlength{\cslentryspacingunit}{\parskip}
% for Pandoc 2.8 to 2.10.1
\newenvironment{cslreferences}%
  {}%
  {\par}
% For Pandoc 2.11+
\newenvironment{CSLReferences}[2] % #1 hanging-ident, #2 entry spacing
 {% don't indent paragraphs
  \setlength{\parindent}{0pt}
  % turn on hanging indent if param 1 is 1
  \ifodd #1
  \let\oldpar\par
  \def\par{\hangindent=\cslhangindent\oldpar}
  \fi
  % set entry spacing
  \setlength{\parskip}{#2\cslentryspacingunit}
 }%
 {}
\usepackage{calc}
\newcommand{\CSLBlock}[1]{#1\hfill\break}
\newcommand{\CSLLeftMargin}[1]{\parbox[t]{\csllabelwidth}{#1}}
\newcommand{\CSLRightInline}[1]{\parbox[t]{\linewidth - \csllabelwidth}{#1}\break}
\newcommand{\CSLIndent}[1]{\hspace{\cslhangindent}#1}

\providecommand{\pandocbounded}[1]{#1}
\usepackage{booktabs}
\usepackage{float}
\usepackage{graphicx}
\usepackage{amsmath}
\usepackage{amsfonts}
\usepackage{amssymb}
\usepackage{bbm}
\usepackage{algorithm}
\usepackage{algpseudocode}
\usepackage{setspace}
\DeclareMathOperator*{\argmin}{argmin}
\DeclareMathOperator*{\argmax}{argmax}
\DeclareMathOperator\diag{diag}
\renewcommand{\algorithmicrequire}{\textbf{Input:}}
\renewcommand{\algorithmicensure}{\textbf{Output:}}

\templatetype{pnasresearcharticle}  % Choose template

\title{Survival driven deconvolution (DeSurv) reveals prognostic and
interpretable cancer subtypes}

\author[a,1,2]{Amber M. Young}
\author[b]{Alisa Yurovsky}
\author[a]{Didong Li}
\author[a,c]{Naim U. Rashid}

  \affil[a]{University of North Carolina at Chapel Hill, Biostatistics,
Street, City, State, Zip}
  \affil[b]{Stony Brook University, Street, City, State, Zip}


% Please give the surname of the lead author for the running footer
\leadauthor{Anonymous}

% Please add here a significance statement to explain the relevance of your work
\significancestatement{Tumor transcriptomes mix malignant and
microenvironmental signals, making it difficult to identify programs
that drive clinical outcomes. Existing deconvolution and matrix
factorization methods discover latent programs but do not ensure
prognostic relevance, while supervised predictors often sacrifice
biological interpretability. We present DeSurv, a survival-supervised
deconvolution framework that integrates nonnegative matrix factorization
with Cox modeling to learn gene programs and their survival associations
jointly. By embedding outcome information into discovery and using
automatic model selection, DeSurv yields clinically relevant,
reproducible programs across cohorts. This advances tumor deconvolution
and provides a general tool for identifying actionable drivers of
disease progression.}


\authorcontributions{Please provide details of author contributions
here.}

\authordeclaration{Please declare any conflict of interest here.}


\correspondingauthor{\textsuperscript{2} To whom correspondence should
be addressed. E-mail:
\href{mailto:ayoung31@live.unc.edu}{\nolinkurl{ayoung31@live.unc.edu}}}

% Keywords are not mandatory, but authors are strongly encouraged to provide them. If provided, please include two to five keywords, separated by the pipe symbol, e.g:
 \keywords{  one |  two |  optional |  optional |  optional  } 

\begin{abstract}
Molecular subtyping in cancer is an ongoing problem that relies on the
identification of robust and replicable gene signatures. While
transcriptomic profiling has revealed recurrent gene expression patterns
in various types of cancer, the prognostic value of these signatures is
typically evaluated in retrospect. This is due to the reliance on
unsupervised learning methods for identifying cell-type-specific signals
and clustering patients into molecular subtypes. Here we present a
Survival-driven Deconvolution tool (deSurv) that integrates bulk
RNA-sequencing data with patient survival information to identify
cell-type-enriched gene signatures associated with prognosis. Applying
deSurv to various cohorts in pancreatic cancer, we uncover prognostic
and biologically interpretable subtypes that reflect the complex
interactions between stroma, tumor, and immune cells in the tumor
microenvironment. Our approach highlights the value of using patient
outcomes during gene signature discovery.
\end{abstract}

\dates{This manuscript was compiled on \today}
\doi{\url{www.pnas.org/cgi/doi/10.1073/pnas.XXXXXXXXXX}}

\begin{document}

% Optional adjustment to line up main text (after abstract) of first page with line numbers, when using both lineno and twocolumn options.
% You should only change this length when you've finalised the article contents.
\verticaladjustment{-2pt}



\maketitle
\thispagestyle{firststyle}
\ifthenelse{\boolean{shortarticle}}{\ifthenelse{\boolean{singlecolumn}}{\abscontentformatted}{\abscontent}}{}

% If your first paragraph (i.e. with the \dropcap) contains a list environment (quote, quotation, theorem, definition, enumerate, itemize...), the line after the list may have some extra indentation. If this is the case, add \parshape=0 to the end of the list environment.

\acknow{Please include your acknowledgments here, set in a single
paragraph. Please do not include any acknowledgments in the Supporting
Information, or anywhere else in the manuscript.}

Molecular subtyping has transformed precision oncology by stratifying
patients into biologically and clinically meaningful groups that inform
prognosis and guide therapy (1--5). Subtyping relies on the
identification of robust biological signals that define subtypes such as
transcriptomic signatures. However, the tumor microenvironment (TME)
contains mixtures of diverse cell types such as malignant, stromal,
immune, and endothelial cells, and disentangling tumor specific signals
from this mixture can be challenging. As such, subtyping pipelines
typically rely on the deconvolution of bulk transcriptomic data or
single-cell analysis to discover distinct cell types and their
corresponding signatures. Downstream, the signatures are evaluated for
clinical relevance such as overall survival or response to treatment.

Separating discovery from validation can risk overfitting and limits
biological and clinical generalizability. Identified cell types may
capture dataset-specific noise rather than reproducible biological
signals, undermining their utility in downstream analyses or therapeutic
targeting (6, 7). Moreover, even when discovered cell types are
biologically valid and reproducible, they may not correspond to the
cellular programs most relevant for predicting or influencing clinical
outcomes (8, 9). Therefore, there is a clear need for integrative
methods that jointly uncover biologically meaningful programs while
directly incorporating clinical endpoints to ensure prognostic
relevance.

However, integrating patient outcomes into the discovery phase is not
straightforward with current technology and methodology. Single-cell
transcriptomics can resolve programs at the cellular level, but cohort
sizes are often too small to support survival analyses. In contrast,
large bulk transcriptomic cohorts with clinical annotations are
well-suited for outcome modeling (10, 11), yet deconvolution is needed
to disentangle overlapping cellular signals. Reference-based
deconvolution methods focus on estimating cell-type proportions from
predefined signatures, which limits the utility of these methods for
discovery of novel programs (12).

Nonnegative matrix factorization (NMF) is widely used in cancer genomics
because its nonnegativity constraints produce biologically
interpretable, additive molecular programs (13--16). Although recent
extensions have incorporated supervision into the factorization, most
target regression or classification rather than time-to-event outcomes.
Two studies have proposed survival-aware NMF formulations (17, 18), but
both integrate the survival objective through the sample-specific
loadings rather than the gene-level programs. This design emphasizes
prediction accuracy but limits the model's ability to restructure or
refine the underlying molecular programs, reducing its value for
biological interpretation and subtype discovery, which are core
objectives in cancer transcriptomics. In addition, neither study
provides a principled approach for hyperparameter selection or model
assessment, and convergence properties are only briefly addressed in one
manuscript. Both works remain unpublished and unreviewed, leaving their
methodological robustness and reproducibility uncertain. These gaps
highlight the need for a rigorously formulated, survival-aware
deconvolution method that jointly estimates interpretable molecular
programs and their prognostic relevance.

Here we present DeSurv, a Survival-supervised Deconvolution framework
that integrates non-negative matrix factorization (NMF) with Cox
proportional hazards modelling. In contrast to fully unsupervised
approaches that evaluate survival associations only after the
factorization, and to existing supervised NMF models that link outcomes
to the subject-level factor loadings, DeSurv integrates survival
information directly into the gene signature matrix. This design ensures
that the discovered transcriptional programs are not only biologically
interpretable but also intrinsically aligned with patient outcomes. To
enhance robustness and reproducibility, DeSurv performs automatic
parameter selection via Bayesian optimization, addressing the
quintessential challenge of rank determination in matrix factorization.

By coupling latent program discovery with direct survival supervision,
DeSurv resolves longstanding challenges in disentangling
tumor--microenvironment interactions and aligns molecular heterogeneity
with clinical outcomes. This unified approach represents a
methodological advance in translational cancer genomics and provides a
general framework for deriving actionable insights from high-dimensional
transcriptomic data.

\hypertarget{results}{%
\section*{Results}\label{results}}
\addcontentsline{toc}{section}{Results}

\hypertarget{model-overview}{%
\subsection*{Model Overview}\label{model-overview}}
\addcontentsline{toc}{subsection}{Model Overview}

We have developed an integrated framework, DeSurv, that couples
Nonnegative Matrix Factorization (NMF) with Cox proportional hazards
regression to identify latent gene-expression programs associated with
patient survival (Figure @ref(fig:fig-schema)). The model takes as input
a bulk expression matrix of \(p\) genes by \(n\) patients (\(X\))
together with corresponding survival times (\(y\)) and censoring
indicators (\(\delta\)) (Figure \ref{fig:schema}A).

DeSurv optimizes a joint objective combining the NMF reconstruction loss
and the Cox model's log-partial likelihood, weighted by a supervision
parameter (\(\alpha\)) that determines the relative contribution of each
term (fig.~\ref{fig:schema}B): \begin{align}
  (1-\alpha)\ &\mathcal{L}_{NMF}(X \approx WH)\nonumber \\ - \alpha\ &\mathcal{L}_{Cox}(Z^\top \beta,y,\delta)
\end{align} When \(\alpha=0\), the method reduces to standard
unsupervised NMF; when \(\alpha>0\), survival information directly
guides the learned factors toward prognostic structure.

Within this framework, the product (\(W^\top X\)) represents
patient-level factor scores (\(Z\)) - the inferred burden of each latent
program across subjects. These factor scores serve as covariates in the
Cox model with linear predictor \(Z^\top \beta\), and their regression
coefficients (\(\beta\)) indicate whether higher activity of a given
program corresponds to improved or reduced survival.

Model training yields gene weights (\(\hat{W}\)), factor loadings
(\(\hat{H}\)), and Cox coefficients (\(\hat{\beta}\)) (Figure
\ref{fig:schema}C), where the inner dimension (\(k\)) specifies the
number of latent factors. Genes with high gene weights in one factor and
low gene weights in all others define the factor-specific signature
genes (Figure \ref{fig:schema}D). By integrating survival supervision
into the factorization, DeSurv not only reconstructs the underlying
expression structure, preserving biological interpretability, but also
guides latent factors to be prognostically informative. Subsequent
analyses can therefore focus on the survival-associated gene programs
(Figure \ref{fig:schema}E).

\begin{figure*}[t]

{\centering \includegraphics[width=\textwidth,height=6in]{../figures/model_schematic_final} 

}

\caption{Overview of the DeSurv framework and data-driven model selection. (A) DeSurv integrates nonnegative matrix factorization (NMF) with survival modeling to learn prognostic gene programs from a gene expression matrix $X$. NMF decomposes $X \approx WH$, where $W$ represents gene programs and $H$ sample loadings; the learned programs $W$ are shared with a Cox proportional hazards model that links factor-derived scores $Z = W^\top X$ to survival outcomes via linear predictor $Z^\top \beta$. A tuning parameter $\alpha$ controls the balance between unsupervised structure learning ($\alpha = 0$) and supervised survival association ($\alpha = 1$). (B) Model complexity ($k$), supervision strength ($\alpha$), and regularization ($\lambda$) are selected via Bayesian optimization using cross-validated concordance index. \label{fig:schema}}\label{fig:fig-schema}
\end{figure*}

\hypertarget{outcome-guided-model-selection-resolves-ambiguity-in-nmf-rank-choice}{%
\subsection*{Outcome-guided model selection resolves ambiguity in NMF
rank
choice}\label{outcome-guided-model-selection-resolves-ambiguity-in-nmf-rank-choice}}
\addcontentsline{toc}{subsection}{Outcome-guided model selection
resolves ambiguity in NMF rank choice}

We examined the problem of selecting the number of latent components
(\(k\)) in nonnegative matrix factorization (NMF) using gene expression
data from pancreatic ductal adenocarcinoma (PDAC) cohorts. These
heterogeneous tumor transcriptomes provide a representative setting in
which to evaluate how commonly used unsupervised rank-selection
heuristics behave in practice.

Across a range of candidate ranks, standard NMF diagnostics yielded
inconsistent guidance (Fig. \ref{fig:bo}A-C). Reconstruction residuals
decreased smoothly with increasing \(k\) and did not exhibit a clear
elbow, a pattern consistent with both relatively small solutions
(\(k \approx 3\)-4) and substantially larger ranks (\(k \approx 6\)-8).
The cophenetic correlation coefficient began to decline at low ranks
(\(k \approx 3\)-4) but continued to fluctuate at higher values without
a distinct transition point. In contrast, mean silhouette width,
evaluated across multiple distance metrics, was highest at very small
ranks (\(k \approx 2\)-3) and decreased monotonically thereafter,
favoring low-dimensional solutions that conflicted with recommendations
based on reconstruction error or cophenetic correlation. Together, these
unsupervised criteria pointed to different and incompatible values of
\(k\), highlighting the ambiguity of rank selection in standard NMF when
applied to PDAC data.

To resolve this ambiguity, we applied DeSurv, which incorporates
survival outcomes directly into the factorization process and evaluates
models using survival-based predictive performance. Using the same PDAC
gene expression data, we assessed model performance across the joint
space of the number of components (\(k\)) and supervision strength
(\(\alpha\)) using cross-validated concordance index (C-index). The
resulting C-index surface summarizes expected predictive performance
across candidate models and enables direct comparison of solutions that
differ in both model complexity and degree of supervision (Fig.
\ref{fig:bo}D).

Model selection was based on standard cross-validation principles.
Rather than selecting the single parameter combination with the highest
predicted C-index, we selected the smallest value of \(k\) whose
predicted performance lay within one standard error of the maximum. This
criterion yielded a stable and parsimonious choice of model rank in the
PDAC data, in contrast to the conflicting recommendations produced by
unsupervised NMF heuristics.

To further evaluate rank recovery under controlled conditions, we
conducted simulation studies in which the true underlying rank was known
(\(k = 3\)). Across repeated simulation replicates, DeSurv consistently
selected the correct rank, producing a concentrated distribution of
selected \(k\) values centered at the true value. In contrast, standard
NMF followed by post hoc Cox modeling (\(\alpha = 0\)) exhibited
substantially greater variability and a systematic tendency toward
under-selection. Together, these results indicate that incorporating
outcome information during model fitting improves the reliability of
rank selection in settings where unsupervised criteria yield conflicting
conclusions.

\begin{figure*}[t]

{\centering \includegraphics[width=6in,height=5.5in]{paper_files/figure-latex/fig-bo-1} 

}

\caption{(A-D) Analyses based on real pancreatic ductal adenocarcinoma (PDAC) gene expression data illustrate the ambiguity of rank selection in standard nonnegative matrix factorization (NMF) and the use of outcome supervision in DeSurv. (A-C) Commonly used unsupervised heuristics for selecting the number of components (k) yield inconsistent conclusions. (A) Reconstruction residuals decrease smoothly with increasing k and do not exhibit a clear elbow; diminishing returns could be inferred at intermediate (k $\approx$ 3-4) or larger (k $\approx$ 6-8) ranks. (B) The cophenetic correlation coefficient, often used to select the largest k prior to a marked loss of clustering stability, begins to decline at low ranks (k $\approx$ 3-4) but continues to fluctuate thereafter, providing no unambiguous selection criterion. (C) Mean silhouette width across multiple distance metrics is highest at small ranks (k $\approx$ 2-3) and decreases monotonically with increasing k, favoring lower-dimensional solutions that conflict with the other criteria.(D) Heatmap of the Gaussian process predicted mean cross-validated concordance index (C-index) from Bayesian optimization over the joint space of the number of components (k) and supervision strength ($\alpha$), computed on the same PDAC data. The predicted performance surface summarizes survival prediction accuracy across parameter settings and illustrates how DeSurv uses outcome information to inform model selection. (E) Results from simulation studies with a known underlying rank (k = 3) showing the distribution of selected k values across repeated replicates. DeSurv more consistently recovers the true rank, yielding a concentrated distribution centered at k = 3, whereas standard NMF with post hoc Cox modeling ($\alpha=0$) exhibits greater variability and a tendency toward under-selection. \label{fig:bo}}\label{fig:fig-bo}
\end{figure*}

\hypertarget{desurv-improves-selection-of-prognostic-gene-signatures}{%
\subsection*{DeSurv improves selection of prognostic gene
signatures}\label{desurv-improves-selection-of-prognostic-gene-signatures}}
\addcontentsline{toc}{subsection}{DeSurv improves selection of
prognostic gene signatures}

To test whether supervision improves the selection of prognostic gene
signatures, we used simulations with known lethal factors and tuned the
number of top genes per factor (\texttt{n\_top}) using BO. We compared
the supervised DeSurv model to the unsupervised \(\alpha=0\) baseline
across simulation regimes. Figure \ref{fig:sim}A shows the distribution
of test C-index values, while Figure \ref{fig:sim}B reports the
precision of recovered prognostic genes (mean across lethal factors).
DeSurv consistently yields higher C-index and improved precision,
indicating that survival supervision helps focus signatures on truly
prognostic genes rather than reconstruction-only structure.

\begin{figure*}[t]

{\centering \includegraphics[width=\textwidth]{paper_files/figure-latex/fig-sim-ntop-scenarios-1} 

}

\caption{Performance comparison between DeSurv and unsupervised NMF ($\alpha = 0$) in simulation. (A) Distribution of cross-validated concordance index shows that DeSurv achieves consistently higher survival prediction performance than the unsupervised baseline. (B) Precision for recovering true underlying gene programs is substantially higher for DeSurv, whereas the unsupervised approach exhibits near-zero precision, indicating poor alignment with the ground-truth factors. \label{fig:sim}}\label{fig:fig-sim-ntop-scenarios}
\end{figure*}

\hypertarget{survival-informed-factorization-reorganizes-transcriptional-structure-in-pancreatic-cancer}{%
\subsection*{Survival-informed factorization reorganizes transcriptional
structure in pancreatic
cancer}\label{survival-informed-factorization-reorganizes-transcriptional-structure-in-pancreatic-cancer}}
\addcontentsline{toc}{subsection}{Survival-informed factorization
reorganizes transcriptional structure in pancreatic cancer}

To assess the biological structure captured by standard nonnegative
matrix factorization (NMF) and DeSurv, we examined the overlap between
factor-specific gene rankings and established PDAC gene programs (Fig.
\ref{fig:bio}A-B). Both approaches recovered recognizable biological
signals; however, the organization of these signals across factors
differed in ways that reflect the objectives of each method.

In the standard NMF solution, factors largely reflected dominant sources
of transcriptional variance. One factor was strongly associated with
exocrine-associated expression and opposed immune and stromal
signatures, consistent with a bulk composition axis separating normal or
differentiated tissue from non-epithelial tumor content. A second factor
aligned with classical tumor identity, while a third aggregated immune,
fibroblast, and extracellular matrix--related programs into a single
microenvironmental component. These patterns indicate that standard NMF
captures major biological variation in bulk PDAC expression data, but
merges distinct microenvironmental states and allocates substantial
model capacity to differentiation-driven signals.

By contrast, DeSurv produced a factorization that emphasized axes of
variation more closely aligned with disease aggressiveness. One factor
corresponded to classical tumor programs opposing basal-like expression,
whereas another isolated an activated microenvironmental state
characterized by immune infiltration and stromal remodeling. A third
factor captured aggressive tumor-intrinsic programs associated with
basal-like biology. Notably, exocrine-associated expression did not
dominate any DeSurv factor, suggesting that survival-informed
optimization deprioritizes differentiation-related variation that is
weakly associated with outcome.

These differences were reflected quantitatively by contrasting the
fraction of expression variance explained by each factor with its
contribution to survival (Fig. \ref{fig:bio}C). In the standard NMF
solution, the factor explaining the largest proportion of
transcriptional variance contributed little to survival, consistent with
its enrichment for exocrine and composition-driven programs. In
contrast, DeSurv concentrated survival signal into a single factor that
explained substantially more variation in outcome despite accounting for
a smaller fraction of expression variance. The remaining DeSurv factors
contributed minimal survival signal, indicating that survival-relevant
information was not diffusely distributed across components.

To further characterize how DeSurv reorganizes the transcriptional
structure learned by standard NMF, we examined the correspondence
between factors derived from the two methods (Fig. \ref{fig:bio}D).
Tumor-intrinsic structure was largely preserved, with one DeSurv factor
showing strong correspondence to the classical tumor--associated NMF
factor. In contrast, the microenvironmental factor identified by
standard NMF mapped primarily to a single DeSurv factor enriched for
activated immune and stromal programs, indicating that DeSurv refines
variance-driven microenvironmental structure into a survival-aligned
axis. Notably, the NMF factor dominated by exocrine-associated
expression did not correspond strongly to any single DeSurv factor,
consistent with the suppression of differentiation- and
composition-driven signals observed in survival-informed factorization.
Together, these results indicate that DeSurv selectively preserves tumor
and microenvironmental structure while reorganizing or deprioritizing
axes of variation that contribute little to survival.

\begin{figure*}[t]

{\centering \includegraphics[width=\textwidth,height=4.5in]{paper_files/figure-latex/fig-bio-1} 

}

\caption{Survival-informed factorization reorganizes transcriptional structure relative to variance-driven NMF. (A) Correlation of standard NMF factor gene rankings with established pancreatic ductal adenocarcinoma (PDAC) gene programs. NMF recovers variance-dominant structure, including an exocrine-associated factor opposing immune and stromal signatures, a classical tumor factor, and a composite microenvironmental factor. (B) Corresponding correlations for DeSurv. DeSurv isolates tumor-intrinsic classical and basal-like programs and resolves an activated immune–stromal microenvironment, while exocrine-associated expression does not dominate any factor. Asterisks indicate significant correlations after multiple testing correction. Only gene lists with correlation > 0.2 included in A and B. (C) Fraction of expression variance explained versus survival contribution for each factor, quantified by the change in partial log-likelihood from univariate Cox models. Standard NMF factors explain substantial variance with minimal survival relevance, whereas DeSurv concentrates survival signal into a single factor despite explaining less expression variance. (D) Pairwise correspondence between NMF and DeSurv factors. Tumor-intrinsic structure is preserved across methods, while variance-driven microenvironmental structure is reorganized into a survival-aligned axis; the exocrine-dominated NMF factor shows no strong one-to-one correspondence. \label{fig:bio}}\label{fig:fig-bio}
\end{figure*}

\hypertarget{desurv-derived-latent-structure-generalizes-to-independent-datasets}{%
\subsection*{DeSurv-derived latent structure generalizes to independent
datasets}\label{desurv-derived-latent-structure-generalizes-to-independent-datasets}}
\addcontentsline{toc}{subsection}{DeSurv-derived latent structure
generalizes to independent datasets}

To assess generalization of DeSurv latent factors, the gene-level
signatures learned in the training cohort were used to compute factor
activity scores in independent validation datasets. For each validation
sample, factor activity was quantified by applying the learned
gene-factor weights to the sample's gene expression profile (\(W^TX\)),
yielding a continuous score for each factor.

For each method, we focused on the factor showing the largest increase
in Cox model log partial likelihood in the training data. Across
validation cohorts, this DeSurv-derived factor exhibited consistent
survival effects, with hazard ratio estimates showing limited
variability and predominantly protective associations (Fig.
\ref{fig:extval}A). In contrast, the NMF factor identified by the same
criterion showed greater heterogeneity across datasets and weaker
survival associations.

When validation samples were pooled and stratified into high- and
low-activity groups based on the DeSurv-derived factor, clear separation
of survival trajectories was observed (Fig. \ref{fig:extval}B). Applying
the same procedure to NMF resulted in weaker survival stratification
(Fig. \ref{fig:extval}C).

Together, these results indicate that DeSurv identifies latent factors
whose associations with survival are preserved across datasets.

\begin{figure*}[t]

{\centering \includegraphics[width=\textwidth]{paper_files/figure-latex/fig-extval-1} 

}

\caption{DeSurv learns prognostic structure that generalizes across independent cohorts. (A) Forest plot summarizing hazard ratios (HRs; 95\% CIs) for the latent factor most predictive of survival in held-out validation datasets. Estimates from DeSurv (blue) are more stable across cohorts than those obtained with standard NMF (red). (B) Kaplan–Meier curves for pooled validation samples stratified into high and low groups based on values of the DeSurv-derived factor providing the strongest survival signal. Median-based stratification yields clear separation of survival trajectories, indicating robust out-of-sample prognostic performance. (C) Corresponding analysis for the NMF-derived factor selected using the same survival-based criterion, showing weaker discrimination between risk groups. \label{fig:extval}}\label{fig:fig-extval}
\end{figure*}

\hypertarget{survival-informed-factorization-identifies-prognostic-structure-across-cancers}{%
\subsection*{Survival-informed factorization identifies prognostic
structure across
cancers}\label{survival-informed-factorization-identifies-prognostic-structure-across-cancers}}
\addcontentsline{toc}{subsection}{Survival-informed factorization
identifies prognostic structure across cancers}

To assess whether the differences observed between standard nonnegative
matrix factorization (NMF) and DeSurv in pancreatic ductal
adenocarcinoma extend beyond a single disease context, we applied both
methods to an independent bladder cancer cohort and evaluated factor
structure, biological interpretation, and survival association using the
same analytical framework (Fig. \ref{fig:bladder}).

In bladder cancer, standard NMF again organized transcriptional
structure primarily around variance-dominant axes (Fig.
\ref{fig:bladder}A). As in PDAC, NMF factors explained substantial
fractions of expression variance but contributed little to survival,
indicating that variance-driven factorization alone does not
preferentially isolate prognostically relevant signals in this setting.
This pattern mirrors the behavior observed in PDAC and suggests that the
limitations of variance-only factorization are not cancer-type specific.

We next asked whether survival-aligned structure learned in PDAC could
transfer across cancer types. Applying the DeSurv model trained in PDAC
directly to bladder cancer samples, the survival-aligned factor retained
prognostic relevance in the external cohort (Fig. \ref{fig:bladder}B).
Kaplan--Meier analysis based on a median split of projected factor
scores demonstrated clear separation of survival curves, despite
differences in tissue context and transcriptional background. This
result indicates that DeSurv learns latent representations that capture
survival-relevant structure shared across cancers.

Together, these results show that survival-informed factorization
separates variance-dominant from survival-relevant structure and that
prognostic signal can transfer across cancer types. By deprioritizing
variance-dominant but prognostically neutral signals and concentrating
outcome-relevant information into a small number of factors, DeSurv
identifies latent structure that is robust across cancers.

\begin{figure*}[t]

{\centering \includegraphics[width=\textwidth]{paper_files/figure-latex/fig-bladder-1} 

}

\caption{Survival-associated structure in bladder cancer is recovered by DeSurv and transfers across tumor types.(A) In bladder cancer, DeSurv identifies latent factors (F1, F2) with strong associations to survival, as measured by Cox partial log-likelihood, despite explaining only a small fraction of total expression variance. In contrast, factors derived from standard NMF explain substantially more variance but exhibit weak survival association. (B) Applying gene-level factor definitions learned in pancreatic ductal adenocarcinoma (PDAC) to bladder cancer samples yields clear separation of survival outcomes when patients are stratified by factor score (median split), with numbers at risk shown below. These results indicate that prognostically relevant structure may be weakly aligned with dominant transcriptional variation and can generalize across cancer types. \label{fig:bladder}}\label{fig:fig-bladder}
\end{figure*}

\hypertarget{discussion}{%
\section*{Discussion}\label{discussion}}
\addcontentsline{toc}{section}{Discussion}

We present DeSurv, a survival-driven deconvolution framework that
integrates nonnegative matrix factorization with Cox proportional
hazards modeling to uncover latent gene-expression programs associated
with patient outcomes. By coupling matrix decomposition with direct
survival supervision, DeSurv targets prognostic structure during
discovery rather than only in post hoc evaluation, and its Bayesian
optimization strategy addresses the ambiguity of rank selection in
standard NMF.

In pancreatic ductal adenocarcinoma (PDAC), survival supervision
reorganized variance-dominant structure into factors more aligned with
outcome. Compared with standard NMF, DeSurv suppressed exocrine- and
composition-driven signals that explained substantial expression
variance but contributed little to survival, and instead concentrated
survival signal into a smaller set of factors. In simulations with known
lethal programs, DeSurv more reliably recovered the true rank and
improved both concordance and precision of prognostic gene signatures,
indicating that outcome-guided learning can sharpen factor
interpretability without sacrificing reconstruction.

The learned factors generalized beyond the training data. Across
independent PDAC cohorts, the most prognostic DeSurv factor showed
consistent hazard ratios and clearer survival separation than the
corresponding NMF factor, indicating that the survival-aligned programs
transfer across datasets. In bladder cancer, DeSurv again separated
variance-dominant from survival-relevant structure, and a PDAC-trained
factor retained prognostic signal when projected into bladder samples.
Together, these results support DeSurv as a general framework for
learning interpretable, outcome-aligned programs in heterogeneous
tumors. Future work will test extensions to additional omics layers,
alternative outcome models, and larger multi-cohort benchmarks to
clarify when outcome supervision offers the greatest benefit.

\hypertarget{materials-and-methods}{%
\section*{Materials and methods}\label{materials-and-methods}}
\addcontentsline{toc}{section}{Materials and methods}

\hypertarget{problem-formulation-and-notation}{%
\subsection{Problem formulation and
notation}\label{problem-formulation-and-notation}}

Let \(X \in \mathbb{R}^{p \times n}_{\ge 0}\) denote the nonnegative
gene expression matrix. DeSurv approximates \(X \approx WH\), where
\(W \in \mathbb{R}^{p\times k}_{\ge0}\) contains nonnegative gene
programs and \(H \in \mathbb{R}^{k\times n}_{\ge0}\) contains
sample-level activations. Additionally, let
\(y,\ \delta \in \mathbb{R}^n\) represent patient survival times and
censoring indicators, respectively. Survival outcomes are modeled
through a Cox proportional hazards model with covariates
\(Z = W^\top X\) and linear predictor \(Z^\top \beta\).

\hypertarget{the-desurv-model}{%
\subsection{The DeSurv Model}\label{the-desurv-model}}

DeSurv integrates Nonnegative Matrix Factorization (NMF) with penalized
Cox regression to identify gene programs associated with patient
survival.

The joint objective is\\
\begin{equation}
\label{eqn:desurv}
\mathcal{L}(W,H,\beta) =
(1-\alpha)\,\mathcal{L}_{\mathrm{NMF}}(W,H)
- \alpha\,\mathcal{L}_{\mathrm{Cox}}(W,\beta),
\end{equation} where \(\mathcal{L}_{\mathrm{NMF}}(W,H)\) is the NMF
reconstruction error and \(\mathcal{L}_{\mathrm{Cox}}(W,\beta)\) is the
elastic-net penalized partial log-likelihood. Optimization proceeds by
alternating updates for \(H\), \(W\), and \(\beta\), using
multiplicative rules for \(H\) (13), projected gradients for \(W\), and
coordinate descent for \(\beta\). Although non-convex, these updates are
shown to converge to a stationary point under mild conditions (SI
Appendix). Complete derivations and algorithmic details are provided in
the SI Appendix.

\hypertarget{hyperparameter-selection-and-cross-validation}{%
\subsection{Hyperparameter selection and
cross-validation}\label{hyperparameter-selection-and-cross-validation}}

Hyperparameters \((k,\alpha,\lambda_H,\lambda,\xi)\) were selected by
maximizing the cross-validated C-index using Bayesian optimization. Each
fold was trained using multiple random initializations, and fold-level
performance was defined as the average C-index across initializations.
For stability, we used a consensus-based initialization for the final
model, aggregating multiple DeSurv runs into a gene--gene co-occurrence
matrix and constructing an initialization \(W_0\) from the resulting
clusters (SI Appendix). Before validation, each column of \(W\) was
truncated to its BO-selected number of top genes (details in SI
Appendix), denoted \(\tilde{W}\). External validation was performed by
first projecting new datasets onto the learned programs via
\(Z = \tilde{W}^\top X_{\text{new}}\) and evaluating survival
associations using C-index and log-rank statistics. To evaluate the
quality of the DeSurv derived gene signatures for subtyping, the new
datasets \(X_new\) were clustered on genes in \(\tilde{W}\), and
survival differences were analyzed for the derived clusters. Further
training, validation, and runtime details appear in the SI Appendix.

\hypertarget{simulation-studies}{%
\subsection{Simulation studies}\label{simulation-studies}}

Simulation studies were conducted to assess recovery of prognostic
latent structure and survival prediction. Gene expression data were
generated from a non-negative factor model \(X=WH\), where gene loadings
\(W\) comprised three gene classes: marker genes, background genes, and
noise genes. Marker genes were simulated to load strongly on a single
factor and weakly on others, background genes to load strongly across
all factors, and noise genes to have uniformly low loadings; each class
was generated from a distinct gamma distribution. Sample-level factor
activities \(H\) were generated from a gamma distribution.

Survival times were generated from an exponential distribution in which
risk depended on marker gene structure through \(X^T\tilde{W}\), where
\(\tilde{W}\) retained marker gene loadings for their corresponding
factors and was zero otherwise; censoring times were generated
independently from an exponential distribution. Each dataset was
analyzed using DeSurv and standard NMF followed by Cox regression on
inferred factors, both tuned using cross-validated concordance index.
Performance was summarized across repeated simulation replicates.

\hypertarget{real-world-datasets}{%
\subsection{Real-world datasets}\label{real-world-datasets}}

We analyzed publicly available RNA-seq and microarray cohorts of
pancreatic ductal adenocarcinoma (PDAC) and bladder cancer with
corresponding overall survival outcomes. Gene expression matrices were
converted to TPM, log-transformed, and filtered to remove low-expression
genes. Survival times and censoring indicators were taken from the
associated clinical annotations. Of the seven PDAC cohorts we
considered, two were used for training (TCGA and CPTAC) and the rest
were used for external validation (Dijk, Moffitt, PACA, Puleo). The
bladder cohort was split into training and validation cohorts via a
70/30 split. To harmonize differences in scale across cohorts, filtered
gene expression data was within-subject rank transformed before model
training. More details about the datasets can be found in the SI
Appendix.

\hypertarget{simulations-benchmarking-and-availability}{%
\subsection{Simulations, Benchmarking, and
Availability}\label{simulations-benchmarking-and-availability}}

An R package has been developed for DeSurv at
github.com/ayoung31/DeSurv. Code and processed data used in this study
are available at github.com/ayoung31/DeSurv-paper.

\showmatmethods
\showacknow

\pnasbreak

\hypertarget{versioning}{%
\section*{Versioning}\label{versioning}}
\addcontentsline{toc}{section}{Versioning}

\begin{verbatim}
## DeSurv package version: HEAD
## DeSurv git branch: installed
## DeSurv git commit: unknown
## Paper git branch: naimedits0125
## Paper git commit: 0c5c01ec10c42c170c996ddb357de71cca733e12
\end{verbatim}

\hypertarget{refs}{}
\begin{CSLReferences}{0}{0}
\leavevmode\vadjust pre{\hypertarget{ref-pareja2016triple}{}}%
\CSLLeftMargin{1. }%
\CSLRightInline{Pareja F, et al. (2016) Triple-negative breast cancer:
The importance of molecular and histologic subtyping, and recognition of
low-grade variants. \emph{NPJ breast cancer} 2(1):1--11.}

\leavevmode\vadjust pre{\hypertarget{ref-dienstmann2018molecular}{}}%
\CSLLeftMargin{2. }%
\CSLRightInline{Dienstmann R, Salazar R, Tabernero J (2018) Molecular
subtypes and the evolution of treatment decisions in metastatic
colorectal cancer. \emph{Am Soc Clin Oncol Educ Book} 38(38):231--8.}

\leavevmode\vadjust pre{\hypertarget{ref-zhou2021clinical}{}}%
\CSLLeftMargin{3. }%
\CSLRightInline{Zhou X, et al. (2021) Clinical impact of molecular
subtyping of pancreatic cancer. \emph{Frontiers in cell and
developmental biology} 9:743908.}

\leavevmode\vadjust pre{\hypertarget{ref-seiler2017impact}{}}%
\CSLLeftMargin{4. }%
\CSLRightInline{Seiler R, et al. (2017) Impact of molecular subtypes in
muscle-invasive bladder cancer on predicting response and survival after
neoadjuvant chemotherapy. \emph{European urology} 72(4):544--554.}

\leavevmode\vadjust pre{\hypertarget{ref-prat2015clinical}{}}%
\CSLLeftMargin{5. }%
\CSLRightInline{Prat A, et al. (2015) Clinical implications of the
intrinsic molecular subtypes of breast cancer. \emph{The Breast}
24:S26--S35.}

\leavevmode\vadjust pre{\hypertarget{ref-Ou2021}{}}%
\CSLLeftMargin{6. }%
\CSLRightInline{Ou F, Michiels S, Shyr Y, Adjei AA, Oberg AL (2021)
\href{https://doi.org/10.1016/j.jtho.2021.01.1616}{Biomarker discovery
and validation: Statistical considerations}. \emph{Journal of Thoracic
Oncology} 16(Suppl 15):S539--S547.}

\leavevmode\vadjust pre{\hypertarget{ref-Planey2016}{}}%
\CSLLeftMargin{7. }%
\CSLRightInline{Planey Catherine\,R, Gevaert O (2016)
\href{https://doi.org/10.1186/s13073-016-0281-4}{CoINcIDE: A framework
for discovery of patient subtypes across multiple datasets}.
\emph{Genome Medicine} 8(1):27.}

\leavevmode\vadjust pre{\hypertarget{ref-Prat2014}{}}%
\CSLLeftMargin{8. }%
\CSLRightInline{Prat A, Pineda E, Adamo B, et\,al. (2014)
\href{https://doi.org/10.1093/jnci/dju152}{Molecular features and
survival outcomes of the intrinsic subtypes in the international breast
cancer study group trial 10‑93}.
\emph{Journal\,of\,the\,National\,Cancer\,Institute} 106(8):dju152.}

\leavevmode\vadjust pre{\hypertarget{ref-ellrott2025classification}{}}%
\CSLLeftMargin{9. }%
\CSLRightInline{Ellrott K, et al. (2025) Classification of non-TCGA
cancer samples to TCGA molecular subtypes using compact feature sets.
\emph{Cancer cell} 43(2):195--212.}

\leavevmode\vadjust pre{\hypertarget{ref-tomczak2015review}{}}%
\CSLLeftMargin{10. }%
\CSLRightInline{Tomczak K, Czerwińska P, Wiznerowicz M (2015) Review the
cancer genome atlas (TCGA): An immeasurable source of knowledge.
\emph{Contemporary Oncology/Wsp{ó}{ł}czesna Onkologia} 2015(1):68--77.}

\leavevmode\vadjust pre{\hypertarget{ref-zhang2019international}{}}%
\CSLLeftMargin{11. }%
\CSLRightInline{Zhang J, et al. (2019) The international cancer genome
consortium data portal. \emph{Nature biotechnology} 37(4):367--369.}

\leavevmode\vadjust pre{\hypertarget{ref-nguyen2024fourteen}{}}%
\CSLLeftMargin{12. }%
\CSLRightInline{Nguyen H, Nguyen H, Tran D, Draghici S, Nguyen T (2024)
Fourteen years of cellular deconvolution: Methodology, applications,
technical evaluation and outstanding challenges. \emph{Nucleic Acids
Research} 52(9):4761--4783.}

\leavevmode\vadjust pre{\hypertarget{ref-lee1999learning}{}}%
\CSLLeftMargin{13. }%
\CSLRightInline{Lee DD, Seung HS (1999) Learning the parts of objects by
non-negative matrix factorization. \emph{nature} 401(6755):788--791.}

\leavevmode\vadjust pre{\hypertarget{ref-Bailey2016}{}}%
\CSLLeftMargin{14. }%
\CSLRightInline{Bailey P, Chang DK, et al. (2016)
\href{https://doi.org/10.1038/nature16965}{Genomic analyses identify
molecular subtypes of pancreatic cancer}. \emph{Nature}
531(7592):47--52.}

\leavevmode\vadjust pre{\hypertarget{ref-moffitt2015virtual}{}}%
\CSLLeftMargin{15. }%
\CSLRightInline{Moffitt RA, et al. (2015) Virtual microdissection
identifies distinct tumor-and stroma-specific subtypes of pancreatic
ductal adenocarcinoma. \emph{Nature genetics} 47(10):1168--1178.}

\leavevmode\vadjust pre{\hypertarget{ref-peng2019novo}{}}%
\CSLLeftMargin{16. }%
\CSLRightInline{Peng XL, Moffitt RA, Torphy RJ, Volmar KE, Yeh JJ (2019)
De novo compartment deconvolution and weight estimation of tumor samples
using DECODER. \emph{Nature communications} 10(1):4729.}

\leavevmode\vadjust pre{\hypertarget{ref-le2025survnmf}{}}%
\CSLLeftMargin{17. }%
\CSLRightInline{Le Goff V, et al. (2025) SurvNMF: Non-negative matrix
factorization supervised for survival data analysis. PhD thesis
(Institut Pasteur Paris; CEA).}

\leavevmode\vadjust pre{\hypertarget{ref-huang2020low}{}}%
\CSLLeftMargin{18. }%
\CSLRightInline{Huang Z, Salama P, Shao W, Zhang J, Huang K (2020)
Low-rank reorganization via proportional hazards non-negative matrix
factorization unveils survival associated gene clusters. \emph{arXiv
preprint arXiv:200803776}.}

\end{CSLReferences}



% Bibliography
% \bibliography{pnas-sample}

\end{document}
