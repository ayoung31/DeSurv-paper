\documentclass[9pt,twocolumn,twoside,]{pnas-new}

% Use the lineno option to display guide line numbers if required.
% Note that the use of elements such as single-column equations
% may affect the guide line number alignment.


\usepackage[T1]{fontenc}
\usepackage[utf8]{inputenc}

% tightlist command for lists without linebreak
\providecommand{\tightlist}{%
  \setlength{\itemsep}{0pt}\setlength{\parskip}{0pt}}


% Pandoc citation processing
%From Pandoc 3.1.8
% definitions for citeproc citations
\NewDocumentCommand\citeproctext{}{}
\NewDocumentCommand\citeproc{mm}{%
  \begingroup\def\citeproctext{#2}\cite{#1}\endgroup}
\makeatletter
 % allow citations to break across lines
 \let\@cite@ofmt\@firstofone
 % avoid brackets around text for \cite:
 \def\@biblabel#1{}
 \def\@cite#1#2{{#1\if@tempswa , #2\fi}}
\makeatother
\newlength{\cslhangindent}
\setlength{\cslhangindent}{1.5em}
\newlength{\csllabelwidth}
\setlength{\csllabelwidth}{3em}
\newenvironment{CSLReferences}[2] % #1 hanging-indent, #2 entry-spacing
 {\begin{list}{}{%
  \setlength{\itemindent}{0pt}
  \setlength{\leftmargin}{0pt}
  \setlength{\parsep}{0pt}
  % turn on hanging indent if param 1 is 1
  \ifodd #1
   \setlength{\leftmargin}{\cslhangindent}
   \setlength{\itemindent}{-1\cslhangindent}
  \fi
  % set entry spacing
  \setlength{\itemsep}{#2\baselineskip}}}
 {\end{list}}
\usepackage{calc}
\newcommand{\CSLBlock}[1]{#1\hfill\break}
\newcommand{\CSLLeftMargin}[1]{\parbox[t]{\csllabelwidth}{#1}}
\newcommand{\CSLRightInline}[1]{\parbox[t]{\linewidth - \csllabelwidth}{#1}\break}
\newcommand{\CSLIndent}[1]{\hspace{\cslhangindent}#1}

\providecommand{\pandocbounded}[1]{#1}
\usepackage{booktabs}
\usepackage{float}
\usepackage{graphicx}
\usepackage{amsmath}
\usepackage{amsfonts}
\usepackage{amssymb}
\usepackage{bbm}
\usepackage{algorithm}
\usepackage{algpseudocode}
\usepackage{setspace}
\DeclareMathOperator*{\argmin}{argmin}
\DeclareMathOperator*{\argmax}{argmax}
\DeclareMathOperator\diag{diag}
\renewcommand{\algorithmicrequire}{\textbf{Input:}}
\renewcommand{\algorithmicensure}{\textbf{Output:}}

\templatetype{pnasresearcharticle}  % Choose template

\title{Survival driven deconvolution (DeSurv) reveals prognostic and
interpretable cancer subtypes}

\author[a,1,2]{Amber M. Young}
\author[b]{Alisa Yurovsky}
\author[a]{Didong Li}
\author[a,c]{Naim U. Rashid}

  \affil[a]{University of North Carolina at Chapel Hill, Biostatistics,
Street, City, State, Zip}
  \affil[b]{Stony Brook University, Street, City, State, Zip}


% Please give the surname of the lead author for the running footer
\leadauthor{Anonymous}

% Please add here a significance statement to explain the relevance of your work
\significancestatement{Tumor transcriptomes mix malignant and
microenvironmental signals, making it difficult to identify programs
that drive clinical outcomes. Existing deconvolution and matrix
factorization methods discover latent programs but do not ensure
prognostic relevance, while supervised predictors often sacrifice
biological interpretability. We present DeSurv, a survival-supervised
deconvolution framework that integrates nonnegative matrix factorization
with Cox modeling to learn gene programs and their survival associations
jointly. By embedding outcome information into discovery and using
automatic model selection, DeSurv yields clinically relevant,
reproducible programs across cohorts. This advances tumor deconvolution
and provides a general tool for identifying actionable drivers of
disease progression.}


\authorcontributions{Please provide details of author contributions
here.}

\authordeclaration{Please declare any conflict of interest here.}


\correspondingauthor{\textsuperscript{2} To whom correspondence should
be addressed. E-mail:
\href{mailto:ayoung31@live.unc.edu}{\nolinkurl{ayoung31@live.unc.edu}}}

% Keywords are not mandatory, but authors are strongly encouraged to provide them. If provided, please include two to five keywords, separated by the pipe symbol, e.g:
 \keywords{  nonnegative matrix factorization |  survival
analysis |  semi-supervised learning |  tumor
deconvolution |  pancreatic cancer  } 

\begin{abstract}
Molecular subtyping in cancer is an ongoing problem that relies on the
identification of robust and replicable gene signatures. While
transcriptomic profiling has revealed recurrent gene expression patterns
in various types of cancer, the prognostic value of these signatures is
typically evaluated in retrospect. This is due to the reliance on
unsupervised learning methods for identifying cell-type-specific signals
and clustering patients into molecular subtypes. Here we present a
Survival-driven Deconvolution tool (DeSurv) that integrates bulk
RNA-sequencing data with patient survival information to identify
cell-type-enriched gene signatures associated with prognosis. Applying
DeSurv to various cohorts in pancreatic cancer, we uncover prognostic
and biologically interpretable subtypes that reflect the complex
interactions between stroma, tumor, and immune cells in the tumor
microenvironment. Our approach highlights the value of using patient
outcomes during gene signature discovery.
\end{abstract}

\dates{This manuscript was compiled on \today}
\doi{\url{www.pnas.org/cgi/doi/10.1073/pnas.XXXXXXXXXX}}

\begin{document}

% Optional adjustment to line up main text (after abstract) of first page with line numbers, when using both lineno and twocolumn options.
% You should only change this length when you've finalised the article contents.
\verticaladjustment{-2pt}



\maketitle
\thispagestyle{firststyle}
\ifthenelse{\boolean{shortarticle}}{\ifthenelse{\boolean{singlecolumn}}{\abscontentformatted}{\abscontent}}{}

% If your first paragraph (i.e. with the \dropcap) contains a list environment (quote, quotation, theorem, definition, enumerate, itemize...), the line after the list may have some extra indentation. If this is the case, add \parshape=0 to the end of the list environment.

\acknow{Please include your acknowledgments here, set in a single
paragraph. Please do not include any acknowledgments in the Supporting
Information, or anywhere else in the manuscript.}

Predicting which cancer patients will respond to therapy or progress
rapidly requires identifying the transcriptional signatures that drive
clinical outcomes (1). Yet in bulk RNA-sequencing, which currently
provides the large clinically annotated cohorts required for survival
modeling, the observed expression profile of each sample reflects
contributions from malignant cells, cancer-associated fibroblasts,
immune infiltrates, and other microenvironmental components (2).
Nonnegative matrix factorization (NMF) has been instrumental in
resolving this mixture into additive, interpretable gene programs
corresponding to recognizable cell types or transcriptional states
(3--5), yielding important biological insights including the
identification of basal-like and classical tumor programs in pancreatic
ductal adenocarcinoma (PDAC) (1, 4) and compartment-specific
deconvolution across 33 cancer types (5).

However, the standard approach discovers these programs through
unsupervised factorization and only then evaluates their clinical
relevance retrospectively (6, 7). Because unsupervised NMF minimizes
reconstruction error, its factors tend to be dominated by the
highest-variance patterns in the data, which may reflect tissue
composition or other outcome-neutral sources rather than prognostic
biology (8, 9). In PDAC, exocrine and tissue-composition signals
dominate expression variance but contribute little to survival
prediction, while lower-variance basal-like and activated stromal
programs carry the strongest prognostic associations (4, 10, 11);
programs that are prognostically relevant but explain modest variance
can be diluted across multiple factors or missed entirely. This
misalignment between variance and prognosis extends beyond PDAC: tumor
purity alone accounts for a large fraction of expression variation
across cancer types (12), yet the objective optimized during
unsupervised discovery (reconstruction error) differs fundamentally from
the criterion used during evaluation (survival association) (9).
Identifying the prognostically relevant subset then requires extensive
downstream filtering that is ad hoc, cohort-specific, and difficult to
reproduce (13). Over a decade of PDAC subtyping efforts proposed between
two and six subtypes (13, 14), ultimately converging on a robust
basal/classical dichotomy only after extensive retrospective evaluation
across independent cohorts (1, 7, 10).

Whether incorporating survival information during factorization improves
generalizability, or risks overfitting to cohort-specific outcome
distributions, remains an open empirical question. Sufficient dimension
reduction theory establishes that response-guided subspace estimation
targets the directions most relevant to the outcome, whereas
variance-maximizing projections can miss outcome-relevant structure
entirely (9, 15). When the outcome depends on cell populations that are
compositionally minor or contribute low variance to bulk expression,
unsupervised methods systematically miss these features, while
outcome-supervised methods can recover them (8, 15, 16). Supervision
during factorization would filter out variance-dominant but
prognostically neutral signals during learning rather than in a separate
post hoc step, but whether this theoretical advantage translates to
improved generalization in the NMF deconvolution setting, where
nonnegative constraints, censored survival outcomes, and cohort-specific
variation introduce additional challenges, has not been empirically
tested.

Here we present DeSurv, a survival-supervised deconvolution framework
that integrates NMF with Cox proportional hazards modeling. The key
architectural choice is where survival supervision enters the
factorization. In DeSurv, factor scores are defined as \(Z = W^\top X\),
so the Cox partial likelihood is a function of the gene program matrix
\(W\) and regression coefficients \(\beta\), and the survival gradient
acts directly on gene programs. Sample-level loadings \(H\) are updated
solely through the reconstruction objective and receive no survival
gradient, preserving their interpretation as mixture coefficients (3,
17) and the biological interpretability of the deconvolution. DeSurv
operates as a semi-supervised method: a parameter \(\alpha\) balances
reconstruction fidelity (\(1-\alpha\)) and survival prediction
(\(\alpha\)). Because \(\alpha\) is selected via cross-validated
concordance, values large enough to overfit to cohort-specific survival
patterns are penalized by poor out-of-sample performance, preventing the
survival term from overwhelming the factorization. Because \(W\) defines
shared transcriptomic programs, new samples can be scored by projection
(\(Z_{\text{new}} = W^\top X_{\text{new}}\)) without requiring their
survival data, a property not shared by methods that route supervision
through \(H\) (18, 19). Hyperparameters, including factorization rank
\(k\) and \(\alpha\), are jointly selected via cross-validated
concordance (Methods).

We evaluate DeSurv in three settings that test the predictions above.
First, in simulations with known latent structure and survival
associations, we show that DeSurv recovers the true factorization rank
and the identity of prognostic programs more reliably than standard NMF,
that its advantage scales with the degree of divergence between variance
and prognosis, and that it vanishes under null conditions where no
survival signal exists. Second, in PDAC, we demonstrate that survival
supervision reorganizes the learned factor structure: DeSurv suppresses
variance-dominant but prognostically neutral signals (e.g., exocrine
content) and concentrates survival association into a smaller set of
biologically interpretable factors aligned with known tumor and
microenvironmental programs. These survival-aligned factors generalize
across independent PDAC cohorts with consistent hazard ratios and
clearer survival separation than their unsupervised counterparts. Third,
we show that a PDAC-trained DeSurv factor retains prognostic signal when
projected into bladder cancer samples, consistent with prior reports
that basal-like transcriptional structure generalizes across epithelial
cancers (20, 21).

\section*{Results}\label{results}
\addcontentsline{toc}{section}{Results}

\subsection*{Model Overview}\label{model-overview}
\addcontentsline{toc}{subsection}{Model Overview}

We developed DeSurv, a survival-supervised deconvolution framework that
jointly optimizes NMF reconstruction and Cox proportional hazards
likelihood (Fig. \ref{fig:schema}; Methods). The survival gradient acts
on the gene program matrix \(W\) but not on the sample loadings \(H\),
directing gene programs toward outcome-relevant structure while
preserving interpretability of sample loadings as mixture coefficients
(3, 17). We evaluated DeSurv in three settings: controlled simulations,
PDAC cohort analysis with external validation, and cross-cancer
transfer.

\begin{figure*}[t]

{\centering \includegraphics[width=\textwidth,height=6in]{../figures/model_schematic_final} 

}

\caption{Overview of the DeSurv framework. (A) DeSurv jointly optimizes NMF reconstruction ($X \approx WH$) and Cox proportional hazards likelihood. The gene program matrix $W$ is shared between objectives: factor scores $Z = W^\top X$ serve as covariates in the Cox model with coefficients $\beta$. A supervision parameter $\alpha$ controls the balance between reconstruction fidelity ($\alpha = 0$, standard NMF) and survival prediction ($\alpha > 0$). (B) The factorization rank ($k$), supervision strength ($\alpha$), and regularization ($\lambda$) are selected via Bayesian optimization over cross-validated concordance index. \label{fig:schema}}\label{fig:fig-schema}
\end{figure*}

\subsection*{Survival supervision clarifies NMF rank
selection}\label{survival-supervision-clarifies-nmf-rank-selection}
\addcontentsline{toc}{subsection}{Survival supervision clarifies NMF
rank selection}

Before examining DeSurv's performance across the three evaluation
settings described above, we address a prerequisite question using both
PDAC expression data and controlled simulations: whether incorporating
clinical outcomes clarifies the well-documented instability of
unsupervised rank selection heuristics (6, 22). Using gene expression
data from the PDAC training cohorts (TCGA (23) and CPTAC (24); Methods),
we first evaluated commonly used unsupervised criteria across a range of
candidate ranks.

Standard NMF diagnostics yielded inconsistent guidance (Fig.
\ref{fig:bo}A-C). Reconstruction residuals decreased smoothly with
increasing \(k\) and did not exhibit a clear elbow, a pattern consistent
with both relatively small solutions (\(k \approx 3\)-4) and
substantially larger ranks (\(k \approx 6\)-8). The cophenetic
correlation coefficient began to decline at low ranks
(\(k \approx 3\)-4) but continued to fluctuate at higher values without
a distinct transition point. In contrast, mean silhouette width was
highest at very small ranks (\(k \approx 2\)-3) and decreased
monotonically thereafter, favoring low-dimensional solutions that
conflicted with the other criteria. Together, these unsupervised
heuristics pointed to incompatible values of \(k\), illustrating the
ambiguity of rank selection in standard NMF.

To address this ambiguity, we applied DeSurv, which incorporates
survival outcomes directly into the factorization and evaluates models
using cross-validated concordance index (C-index). The resulting C-index
surface across the joint space of factorization rank (\(k\)) and
supervision strength (\(\alpha\)) identified a well-defined optimum, in
contrast to the conflicting recommendations produced by unsupervised
heuristics (Fig. \ref{fig:bo}D). Model selection followed the
one-standard-error rule: we selected the smallest \(k\) whose predicted
performance lay within one standard error of the maximum, yielding a
parsimonious choice. Bayesian optimization selected \(k = 3\) and
\(\alpha = 0.3339356\) (C-index \(0.655\); 1-SE rule; \(n = 273\)
patients, \(139\) events).

To further evaluate rank recovery under controlled conditions, we
conducted simulation studies in which the true underlying rank was known
(\(k = 3\)). DeSurv consistently selected the correct rank, producing a
concentrated distribution of selected \(k\) values centered at the true
value (Fig. \ref{fig:bo}E). In contrast, standard NMF followed by post
hoc Cox modeling (\(\alpha = 0\)) exhibited substantially greater
variability and a systematic tendency toward under-selection. These
results support the prediction that incorporating outcome information
into model selection improves recovery of the true factorization rank.

\begin{figure*}[t]

{\centering \includegraphics[width=6in,height=5.5in]{/work/users/a/y/ayoung31/DeSurv-paper/paper/paper_files/figure-latex/fig-bo-1} 

}

\caption{(A--D) Analyses based on pancreatic ductal adenocarcinoma (PDAC) gene expression data from TCGA and CPTAC cohorts. (A--C) Standard unsupervised rank selection heuristics yield inconsistent guidance for selecting the factorization rank $k$. (A) Reconstruction residuals as a function of $k$. (B) Cophenetic correlation coefficient as a function of $k$. (C) Mean silhouette width across multiple distance metrics as a function of $k$. (D) Gaussian process predicted mean cross-validated concordance index (C-index) from Bayesian optimization over the joint space of factorization rank ($k$) and supervision strength ($\alpha$). (E) Simulation studies with known rank ($k = 3$): distribution of selected $k$ values across repeated replicates for DeSurv versus standard NMF with post hoc Cox modeling ($\alpha = 0$). \label{fig:bo}}\label{fig:fig-bo}
\end{figure*}

\subsection*{DeSurv recovers prognostic gene programs when variance and
prognosis
diverge}\label{desurv-recovers-prognostic-gene-programs-when-variance-and-prognosis-diverge}
\addcontentsline{toc}{subsection}{DeSurv recovers prognostic gene
programs when variance and prognosis diverge}

Sufficient dimension reduction theory predicts that outcome-guided
subspace estimation recovers directions most relevant to the response
(9, 15), but this prediction has not been evaluated in the NMF
deconvolution setting. To test whether it holds under nonnegative
constraints with survival outcomes, we designed simulations with known
ground-truth latent structure. Simulated expression matrices were
generated from a nonnegative factor model in which prognostic gene
programs explained low variance relative to outcome-neutral background
signals, with survival times generated from marker gene expression
(\(p = 3{,}000\) genes, \(n = 200\) samples, true \(k = 3\); Methods).
We compared DeSurv to standard NMF (\(\alpha=0\)) followed by post hoc
Cox regression, with both methods tuned via cross-validated concordance
index using Bayesian optimization.

In the primary scenario, where prognostic programs explained low
variance relative to outcome-neutral programs, DeSurv consistently
achieved higher C-index and substantially improved precision---the
fraction of genes in a learned factor that belong to a true prognostic
gene program---for recovering the true prognostic gene programs (Fig.
\ref{fig:sim}A-B). The unsupervised baseline exhibited near-zero
precision, indicating that variance-driven factorization failed to
concentrate on the genes that actually drove survival. This result
supports the prediction that outcome-guided learning recovers prognostic
programs more reliably when variance and prognosis diverge. Across 100
replicates, DeSurv achieved median C-index \(\textrm{[TODO]}\) versus
\(\textrm{[TODO]}\) for standard NMF (\(\Delta = \textrm{[TODO]}\);
\(p = 3{,}000\) genes, \(n = 200\) samples, true \(k = 3\)).

To verify that these gains reflect genuine signal recovery rather than
overfitting, we repeated the analysis under two additional simulation
scenarios (SI Appendix, Fig. S2). In a null scenario where survival
times were generated independently of gene expression (\(\beta = 0\)),
DeSurv defaulted to standard NMF: Bayesian optimization selected low
supervision strength (\(\alpha\)), and both methods yielded C-index
values near 0.5. This indicates that the semi-supervised design does not
impose spurious prognostic structure when none exists. In a mixed
scenario where survival depended on both factor-specific marker genes
and shared background genes, creating partial overlap between
variance-dominant and prognostically relevant structure, DeSurv's
advantage was present but attenuated relative to the primary scenario.
Together, these three scenarios establish that DeSurv's benefit scales
with the degree of divergence between variance and prognosis: largest
when prognostic programs explain low variance (primary scenario),
moderate when variance and prognosis partially overlap (mixed), and
absent when no survival signal exists (null).

\begin{figure*}[t]

{\centering \includegraphics[width=\textwidth]{/work/users/a/y/ayoung31/DeSurv-paper/paper/paper_files/figure-latex/fig-sim-ntop-scenarios-1} 

}

\caption{Performance comparison between DeSurv and standard NMF ($\alpha = 0$) in the primary simulation scenario ($p = 3{,}000$ genes, $n = 200$ samples, true $k = 3$, 100 replicates), where prognostic programs explain low variance relative to outcome-neutral programs. Both methods were tuned via Bayesian optimization over cross-validated concordance index. (A) Test-set concordance index across simulation replicates. (B) Precision (fraction of genes in a learned factor belonging to a true prognostic gene program) across simulation replicates. Each point represents one replicate. \label{fig:sim}}\label{fig:fig-sim-ntop-scenarios}
\end{figure*}

\subsection*{Survival supervision reorganizes the learned factor
structure in
PDAC}\label{survival-supervision-reorganizes-the-learned-factor-structure-in-pdac}
\addcontentsline{toc}{subsection}{Survival supervision reorganizes the
learned factor structure in PDAC}

The dominant source of expression variance in PDAC, exocrine content, is
not the dominant source of prognostic signal; DeSurv reorganizes the
learned factor structure accordingly.

To directly test whether survival supervision reorganizes the learned
factor structure, as predicted by the simulation results above, we
examined the overlap between factor-specific gene rankings and
established PDAC gene programs (Fig. \ref{fig:bio}A-B). Bayesian
optimization selected \(k = 3\) and \(\alpha = 0.33\) for DeSurv in the
PDAC training cohorts (TCGA and CPTAC). To enable a direct comparison of
how each method organizes the same number of factors, we also fit
standard NMF at the same rank (\(k = 3\)); results for standard NMF at
independently selected ranks (\(k = 5\) via elbow detection, \(k = 7\)
via cross-validated concordance at \(\alpha = 0\)) are presented in the
SI Appendix.

DeSurv produced a factorization in which each factor aligned with a
distinct, prognostically relevant biological program (Fig.
\ref{fig:bio}A). One factor was enriched for classical tumor programs
relative to basal-like expression, a second isolated an activated
microenvironmental state characterized by immune infiltration and
stromal remodeling, and a third captured aggressive tumor-intrinsic
programs associated with basal-like biology. Notably,
exocrine-associated expression did not dominate any DeSurv factor,
suggesting that survival supervision deprioritizes
differentiation-related variation that is weakly associated with
outcome.

At the same rank, standard NMF organized the three factors around
dominant sources of transcriptional variance rather than prognostic
biology (Fig. \ref{fig:bio}B). One factor was strongly associated with
exocrine expression and negatively correlated with immune and stromal
signatures, consistent with a bulk composition axis separating normal or
differentiated tissue from non-epithelial tumor content. A second factor
aligned with classical tumor identity, while a third aggregated immune,
fibroblast, and extracellular matrix-related programs into a single
composite microenvironmental factor. These patterns indicate that, given
the same number of factors, standard NMF allocates substantial model
capacity to differentiation-driven signals and merges distinct
microenvironmental states that DeSurv separates.

We quantified these differences by contrasting the fraction of
expression variance explained by each factor with its contribution to
survival (Fig. \ref{fig:bio}C). The NMF factor explaining the largest
proportion of transcriptional variance contributed little to survival,
consistent with its enrichment for exocrine and composition-driven
programs. In contrast, DeSurv concentrated survival signal into a single
factor that explained substantially more survival association despite
accounting for a smaller fraction of expression variance. Specifically,
the highest-variance NMF factor explained \(\textrm{[TODO]}\)\% of
expression variance but contributed \(\Delta\ell = \textrm{[TODO]}\) to
survival, whereas the DeSurv factor with the largest survival
contribution explained \(\textrm{[TODO]}\)\% of variance with
\(\Delta\ell = \textrm{[TODO]}\). The remaining DeSurv factors
contributed minimal survival signal, indicating that survival-relevant
information was not diffusely distributed across factors. This result
directly illustrates the misalignment between variance and prognosis
anticipated by sufficient dimension reduction theory (9) and observed
empirically in tumor purity analyses (12): the factors that explain the
most transcriptomic variation need not be the factors most associated
with patient outcomes. The observed pattern in PDAC, where the
highest-variance factor shows minimal survival association while other
factors show partial overlap between variance and prognosis, is
consistent with the intermediate regime between the primary and mixed
simulation scenarios, where DeSurv's advantage was largest.

To further characterize how the learned factor structure differs between
DeSurv and standard NMF, we examined the correspondence between factors
derived from the two methods (Fig. \ref{fig:bio}D). Tumor-intrinsic
structure was largely preserved, with one DeSurv factor showing strong
correspondence to the classical tumor-associated NMF factor. The
microenvironmental factor identified by standard NMF mapped primarily to
a single DeSurv factor enriched for activated immune and stromal
programs, indicating that DeSurv refines variance-driven
microenvironmental structure into a survival-aligned axis. Notably, the
NMF factor dominated by exocrine expression did not correspond strongly
to any single DeSurv factor, consistent with the suppression of
differentiation- and composition-driven signals under survival
supervision. Together, these results demonstrate that DeSurv selectively
preserves tumor and microenvironmental structure while reorganizing or
deprioritizing patterns of variation that contribute little to survival.

\begin{figure*}[t]

{\centering \includegraphics[width=\textwidth,height=4.5in]{/work/users/a/y/ayoung31/DeSurv-paper/paper/paper_files/figure-latex/fig-bio-1} 

}

\caption{Survival supervision reorganizes the learned factor structure relative to standard NMF in PDAC training data (TCGA and CPTAC). Both methods are compared at factorization rank $k = 3$, selected by DeSurv's Bayesian optimization; standard NMF at independently selected ranks is shown in SI Appendix. (A) Correlation of DeSurv factor gene rankings with established PDAC gene programs. (B) Corresponding correlations for standard NMF factors. Asterisks indicate significant correlations after multiple testing correction; only gene programs with $|r| > 0.2$ shown. (C) Fraction of expression variance explained versus survival contribution for each factor, quantified by the change in partial log-likelihood from univariate Cox models. (D) Pairwise correlation between NMF and DeSurv factor gene rankings, showing which biological programs are preserved, reorganized, or suppressed under survival supervision. \label{fig:bio}}\label{fig:fig-bio}
\end{figure*}

\subsection*{Survival-aligned programs generalize across independent
PDAC
cohorts}\label{survival-aligned-programs-generalize-across-independent-pdac-cohorts}
\addcontentsline{toc}{subsection}{Survival-aligned programs generalize
across independent PDAC cohorts}

Having shown that DeSurv reorganizes the learned factor structure in the
training data, we next tested whether this learned factor structure
generalizes to independent cohorts. If outcome-guided subspaces capture
biologically reproducible rather than noise-driven structure, they
should transfer more readily across datasets (9, 15). We evaluated
whether DeSurv factors maintain their prognostic associations in
independent PDAC cohorts. We computed factor scores in each validation
dataset by projecting the gene programs learned in the training cohort
(\(Z = W^\top X_{\text{new}}\)), yielding a continuous score for each
factor per sample.

For each method, we focused on the factor showing the largest increase
in Cox model log partial likelihood in the training data. Across five
independent PDAC cohorts (\(n = \textrm{[TODO]}\)), the DeSurv-derived
factor exhibited consistent survival effects, with hazard ratio
estimates showing limited variability and predominantly protective
associations (pooled HR \(\textrm{[TODO]}\); 95\% CI
\(\textrm{[TODO]}\); log-rank \(P = \textrm{[TODO]}\); Fig.
\ref{fig:extval}A). In contrast, the NMF factor identified by the same
criterion showed greater heterogeneity across datasets and weaker
survival associations, consistent with the expectation that
variance-driven programs capture cohort-specific variation that does not
transfer.

We pooled validation samples and stratified them into high- and
low-score groups based on a median split of the DeSurv-derived factor,
observing clear separation of survival trajectories (Fig.
\ref{fig:extval}B). The same procedure applied to NMF yielded weaker
survival stratification (Fig. \ref{fig:extval}C). These results support
the prediction that outcome-aligned programs capture biology that
generalizes, whereas variance-driven programs may include
cohort-specific signals that attenuate across datasets.

\begin{figure*}[t]

{\centering \includegraphics[width=\textwidth]{/work/users/a/y/ayoung31/DeSurv-paper/paper/paper_files/figure-latex/fig-extval-1} 

}

\caption{External validation of DeSurv and standard NMF prognostic factors in independent PDAC cohorts, both at factorization rank $k = 3$. (A) Forest plot of hazard ratios (HRs; 95\% CIs) for the factor with the largest training-set Cox partial log-likelihood contribution, evaluated in five held-out PDAC cohorts. DeSurv (blue) and standard NMF (red). (B) Kaplan--Meier curves for pooled validation samples stratified by median split of the DeSurv-derived factor score. (C) Corresponding median-split stratification for the standard NMF-derived factor. \label{fig:extval}}\label{fig:fig-extval}
\end{figure*}

\subsection*{A PDAC-trained program retains prognostic signal in bladder
cancer}\label{a-pdac-trained-program-retains-prognostic-signal-in-bladder-cancer}
\addcontentsline{toc}{subsection}{A PDAC-trained program retains
prognostic signal in bladder cancer}

A PDAC-trained DeSurv program retains prognostic relevance in bladder
cancer, consistent with shared basal-like transcriptional biology across
epithelial cancers.

Finally, we tested whether the misalignment between variance and
prognosis extends beyond PDAC and whether survival-aligned programs
transfer across cancer types, a prediction motivated by prior evidence
that basal-like transcriptional structure generalizes across epithelial
cancers (20, 21). We first applied both standard NMF and DeSurv to an
independent bladder cancer cohort and evaluated factor structure and
survival association (Fig. \ref{fig:bladder}A). We then asked whether
the DeSurv model trained in PDAC retains prognostic relevance when
projected to bladder cancer samples (Fig. \ref{fig:bladder}B).

In bladder cancer, standard NMF again organized the learned factor
structure primarily around variance-dominant patterns (Fig.
\ref{fig:bladder}A). As in PDAC, NMF factors explained substantial
fractions of expression variance but contributed little to survival,
replicating the misalignment between variance and prognosis observed in
PDAC.

Separately, we projected the DeSurv model trained in PDAC directly onto
bladder cancer samples to test cross-cancer transfer. The
survival-aligned factor retained prognostic relevance in the external
cohort (Fig. \ref{fig:bladder}B). Kaplan-Meier analysis based on a
median split of projected factor scores demonstrated clear separation of
survival curves, despite differences in tissue context and
transcriptional background (\(n = \textrm{[TODO]}\), \(\textrm{[TODO]}\)
events; log-rank \(P = \textrm{[TODO]}\); HR \(\textrm{[TODO]}\); 95\%
CI \(\textrm{[TODO]}\)). This finding suggests that survival supervision
captures cross-cancer biology that unsupervised methods may distribute
across multiple outcome-neutral factors.

Together, these results suggest that survival supervision separates
variance-dominant from survival-relevant structure and that this
separation can transfer across cancer types, providing initial evidence
that outcome-guided dimension reduction targets different subspaces than
variance-driven reduction.

\begin{figure*}[t]

{\centering \includegraphics[width=\textwidth]{/work/users/a/y/ayoung31/DeSurv-paper/paper/paper_files/figure-latex/fig-bladder-1} 

}

\caption{Survival-associated factor structure in bladder cancer and cross-cancer transfer from PDAC. (A) Fraction of expression variance explained versus survival contribution (change in Cox partial log-likelihood) for each factor in bladder cancer, comparing DeSurv and standard NMF. (B) Kaplan--Meier curves for bladder cancer samples stratified by median split of projected factor scores from a PDAC-trained DeSurv model, with numbers at risk shown below. \label{fig:bladder}}\label{fig:fig-bladder}
\end{figure*}

\section*{Discussion}\label{discussion}
\addcontentsline{toc}{section}{Discussion}

We have shown that incorporating survival information during NMF-based
deconvolution reorganizes the learned factor structure, concentrating
prognostic signal into fewer, more interpretable gene programs while
suppressing variance-dominant but outcome-neutral structure. This
reorganization produces a more parsimonious representation of
prognostically relevant biology: in PDAC, DeSurv achieved with three
factors the same cross-validated concordance that standard NMF required
seven to eleven factors to match, and even at each method's unrestricted
optimum, DeSurv attained a higher concordance with fewer factors. The
advantage is most pronounced at low factorization ranks, where standard
NMF allocates its limited representational capacity to variance-dominant
signals such as exocrine content, leaving little room for prognostic
programs to emerge as distinct factors; survival supervision corrects
this allocation, enabling recovery of basal-like and microenvironmental
programs even at \(k = 3\). In simulations, DeSurv recovered the true
factorization rank and the identity of prognostic programs more reliably
than standard NMF, with the largest advantage occurring when prognostic
programs explained modest variance relative to outcome-neutral signals.
In PDAC, DeSurv isolated tumor-intrinsic and microenvironmental programs
with clear survival associations, and these factors generalized across
independent PDAC cohorts. For the dominant DeSurv program, prognostic
signal transferred to bladder cancer, supporting the hypothesis that
outcome-guided learning captures biology that generalizes more readily
than variance-driven structure.

In PDAC, DeSurv recapitulated known tumor and microenvironmental
programs, including the basal-like/classical distinction and activated
versus normal stromal states. The methodological contribution lies not
in the identity of these programs---which have been established through
virtual microdissection (4), experimental microdissection (25), and
unsupervised deconvolution (5)---but in how they are recovered. DeSurv
discovers these programs de novo, without requiring pre-specified
signatures or post hoc filtering, and directly quantifies their survival
associations during the factorization. The resulting factors align with
the consensus that has emerged from over a decade of PDAC subtyping
efforts (1, 4, 7, 13), but are identified through a single optimization
rather than iterative retrospective evaluation. The cross-cancer
transfer result, in which a PDAC-trained factor stratified bladder
cancer survival, is consistent with prior evidence that basal-like
transcriptional programs are shared across epithelial cancers (20, 21)
and suggests that survival supervision captures this shared biology more
directly than variance-driven approaches.

The semi-supervised design of DeSurv reflects a deliberate balance
between biological completeness and clinical relevance. The
reconstruction term (weight \(1-\alpha\)) penalizes deviation from the
observed expression data; the survival term (weight \(\alpha\)) directs
gene programs toward outcome-relevant structure. This balance interacts
with rank selection: survival supervision enables parsimonious
factorizations by concentrating prognostic signal into fewer factors, so
that the one-standard-error rule applied to cross-validated concordance
selects smaller ranks than would be chosen by unsupervised criteria or
by optimizing concordance without supervision. In PDAC, the
cross-validated concordance surface was relatively flat across
\(k = 3\)--12 for DeSurv, indicating that three supervised factors
captured nearly as much prognostic information as twelve; for standard
NMF, concordance increased steadily from \(k = 2\) through
\(k = 7\)--11, reflecting the need for additional factors to distribute
prognostic signal that supervision would have concentrated. Our
simulations provide further guidance on when this tradeoff is most
beneficial: DeSurv's advantage is greatest when prognostic programs
explain modest variance relative to outcome-neutral signals and vanishes
under null conditions where no survival signal exists. DeSurv is
therefore not intended to replace unsupervised NMF in all settings. When
the goal is exploratory discovery without clinical endpoints, when
survival data is sparse or unreliable, or when the application requires
comprehensive biological characterization rather than prognostic
stratification, unsupervised methods remain appropriate. The \(\alpha\)
parameter, tuned via Bayesian optimization, lets the data determine the
appropriate balance for a given cancer type and cohort size, and the
\(\alpha = 0\) endpoint recovers standard NMF as a special case.

Several limitations should be noted. First, DeSurv assumes Cox
proportional hazards, which may not hold in settings where treatment
effects are delayed or non-proportional, such as immunotherapy response;
extensions to more flexible survival models (e.g., accelerated failure
time models or neural network-based hazard functions) are a natural
direction. Second, the computational cost of Bayesian optimization over
cross-validated concordance exceeds that of standard NMF and may limit
scalability to very large cohorts or rapid iterative analyses; however,
hyperparameter selection is performed once per dataset and the final
model can then be applied to arbitrarily many validation cohorts by
projection. Third, while the cross-cancer transfer result is
encouraging, we tested only one transfer pair (PDAC to bladder), and
broader benchmarking across cancer types is needed to establish
generality. Fourth, DeSurv's convergence guarantee (SI Appendix) applies
to an idealized algorithm; the implementation includes practical
modifications (backtracking, gradient clamping) that depart from the
theoretical analysis, though empirically these do not affect convergence
behavior. Fifth, the supervised factorization is optimized with respect
to a specific clinical endpoint, and the resulting molecular programs
may differ under alternative outcome definitions (e.g., progression-free
vs.~overall survival). Moreover, because supervision incorporates
observed survival, which reflects both tumor biology and treatment
received, the learned factors may implicitly encode treatment-response
associations present in the training cohort. The Cox component can
incorporate additional clinical covariates to adjust for known
confounders during optimization, partially mitigating this concern,
though residual confounding from unmeasured factors remains possible.
This limits transportability to settings with substantially different
treatment landscapes and means the biological interpretation of
supervised factors should be understood as conditional on the
therapeutic context in which they were derived. Finally, as with any
supervised method, DeSurv's value depends on the quality of the outcome
data: heavily censored, short follow-up, or misannotated survival
information will degrade the learned programs. The nested
cross-validation framework mitigates overfitting to training labels, but
cannot compensate for systematic outcome misspecification.

More broadly, the principle instantiated by DeSurv---that outcome-guided
dimensionality reduction targets different subspaces than
variance-driven reduction---extends beyond cancer genomics. Sufficient
dimension reduction theory (9) and the information bottleneck framework
(26) both predict that supervised compression retains outcome-relevant
structure while discarding nuisance variation. DeSurv realizes this
principle within the specific constraints of NMF deconvolution, where
nonnegativity preserves biological interpretability and the
factorization structure enables single-sample scoring. The success of
this approach in PDAC and bladder cancer suggests that analogous
frameworks may be valuable in other settings where high-dimensional
measurements contain both signal and nuisance variation---including
multi-omics integration, spatial transcriptomics, and electronic health
records. Extending DeSurv to these domains, and to alternative outcome
models beyond Cox regression, is a natural next step. As single-cell
cohorts with clinical annotation grow in size, direct survival modeling
at cellular resolution may complement deconvolution-based approaches,
though the cohort sizes required for stable survival analysis remain
available primarily in bulk expression data.

\section*{Materials and methods}\label{materials-and-methods}
\addcontentsline{toc}{section}{Materials and methods}

\subsection{Problem formulation and
notation}\label{problem-formulation-and-notation}

Let \(X \in \mathbb{R}^{p \times n}_{\ge 0}\) denote the nonnegative
gene expression matrix. DeSurv approximates \(X \approx WH\), where
\(W \in \mathbb{R}^{p\times k}_{\ge0}\) contains nonnegative gene
programs and \(H \in \mathbb{R}^{k\times n}_{\ge0}\) contains sample
loadings. Additionally, let \(y \in \mathbb{R}^{n}_{>0}\) denote patient
survival times, \(\delta \in \{0,1\}^{n}\) the censoring indicators
(\(\delta_i = 1\) if the event is observed), and
\(\beta \in \mathbb{R}^{k}\) the Cox regression coefficients. Survival
outcomes are modeled through a Cox proportional hazards model with
factor scores \(Z = W^\top X \in \mathbb{R}^{k \times n}\) and linear
predictor \(Z^\top \beta\).

\subsection{The DeSurv Model}\label{the-desurv-model}

DeSurv integrates Nonnegative Matrix Factorization (NMF) with penalized
Cox regression to identify gene programs associated with patient
survival. The method operates in a semi-supervised framework: the
reconstruction loss penalizes deviation from the observed expression
data, while the survival term directs gene programs toward
outcome-relevant structure. The parameter \(\alpha \in [0,1]\) controls
the relative contribution of each objective.

The joint objective is \begin{equation}
\label{eqn:desurv}
\mathcal{L}(W,H,\beta) =
(1-\alpha)\,\mathcal{L}_{\mathrm{NMF}}(W,H)
- \alpha\,\mathcal{L}_{\mathrm{Cox}}(W,\beta),
\end{equation} where \(\mathcal{L}_{\mathrm{NMF}}(W,H)\) is the NMF
reconstruction error and \(\mathcal{L}_{\mathrm{Cox}}(W,\beta)\) is the
elastic-net penalized partial log-likelihood. A critical architectural
choice is where survival supervision enters the factorization. Because
factor scores are defined as \(Z = W^\top X\), the Cox partial
likelihood \(\mathcal{L}_{\mathrm{Cox}}\) is a direct function of \(W\)
and \(\beta\), and the survival gradient acts explicitly on the gene
program matrix \(W\). The sample-level loadings \(H\) enter only through
the reconstruction term and receive no survival gradient, preserving
their interpretation as mixture coefficients (3, 17). When
\(\alpha = 0\), DeSurv reduces to standard unsupervised NMF. The Cox
component also accommodates additional sample-level covariates (e.g.,
tumor stage or grade) alongside \(Z\), enabling adjustment for known
prognostic factors during optimization.

Optimization proceeds by alternating updates for \(H\), \(W\), and
\(\beta\), using multiplicative rules for \(H\) (3), projected gradients
for \(W\), and coordinate descent for \(\beta\). Although non-convex,
these updates are shown to converge to a stationary point under mild
conditions (SI Appendix). Complete derivations and algorithmic details
are provided in the SI Appendix.

\subsection{Hyperparameter selection and
cross-validation}\label{hyperparameter-selection-and-cross-validation}

Hyperparameters \((k,\alpha,\lambda_H,\lambda,\xi)\) were selected by
maximizing the cross-validated C-index using Bayesian optimization, with
final rank \(k\) chosen by the one-standard-error rule (the smallest
\(k\) whose predicted performance lay within one standard error of the
maximum). All model selection was performed entirely within training
data using nested cross-validation; no validation cohort data were used
during any stage of tuning. Because \(\alpha\) is selected via
cross-validated concordance, values large enough to overfit to
cohort-specific survival patterns are penalized by poor out-of-sample
performance, preventing the survival term from overwhelming the
factorization. Each fold was trained using multiple random
initializations, and fold-level performance was defined as the average
C-index across initializations. For stability, we used a consensus-based
initialization for the final model, aggregating multiple DeSurv runs
into a gene-gene co-occurrence matrix and constructing an initialization
\(W_0\) from the resulting clusters (SI Appendix). Before validation,
each column of \(W\) was truncated to its BO-selected number of top
genes (details in SI Appendix), denoted \(\tilde{W}\).

External validation was performed by projecting new datasets onto the
learned programs via \(Z = \tilde{W}^\top X_{\text{new}}\) and
evaluating survival associations using C-index and log-rank statistics.
Because the learned \(W\) is fixed at training time and validation
samples are scored by simple projection, no retraining or access to
validation survival data is required. Reported external hazard ratios
and log-rank statistics therefore reflect purely out-of-sample
generalization. To evaluate the quality of DeSurv-derived gene
signatures for subtyping, new datasets \(X_{\text{new}}\) were clustered
on genes in \(\tilde{W}\), and survival differences were analyzed for
the derived clusters. Further training, validation, and runtime details
appear in the SI Appendix.

\subsection{Simulation studies}\label{simulation-studies}

Simulation studies were conducted to assess recovery of prognostic
latent structure and survival prediction under controlled conditions.
Gene expression data were generated from a nonnegative factor model
\(X=WH\), where gene loadings \(W\) comprised three gene classes: marker
genes, background genes, and noise genes. Marker genes were simulated to
load strongly on a single factor and weakly on others, background genes
to load strongly across all factors, and noise genes to have uniformly
low loadings; each class was generated from a distinct gamma
distribution. Sample-level factor activities \(H\) were generated from a
gamma distribution.

Survival times were generated from an exponential distribution in which
risk depended on marker gene expression through \(X^T\tilde{W}\), where
\(\tilde{W}\) retained marker gene loadings for their corresponding
factors and was zero otherwise; censoring times were generated
independently from an exponential distribution. Each dataset was
analyzed using DeSurv and standard NMF followed by Cox regression on
inferred factors, both tuned using cross-validated concordance index via
Bayesian optimization. Performance was summarized across repeated
simulation replicates. We considered three simulation scenarios: a
primary scenario where prognostic programs explain low variance relative
to outcome-neutral signals, a null scenario with no survival signal, and
a mixed scenario where prognostic and variance-dominant programs
partially overlap.

\subsection{Real-world datasets}\label{real-world-datasets}

We analyzed publicly available RNA-seq and microarray cohorts of
pancreatic ductal adenocarcinoma (PDAC) and bladder cancer with
corresponding overall survival outcomes. Gene expression matrices were
converted to TPM, log-transformed, and filtered to remove low-expression
genes. Survival times and censoring indicators were taken from the
associated clinical annotations. Of the seven PDAC cohorts we
considered, two were used for training (TCGA and CPTAC) and the rest
were used for external validation (Dijk, Moffitt, PACA, Puleo). The
bladder cohort was split into training and validation cohorts via a
70/30 split. To harmonize differences in scale across cohorts, filtered
gene expression data was within-subject rank transformed before model
training. More details about the datasets can be found in the SI
Appendix.

\subsection{Software and availability}\label{software-and-availability}

An R package implementing DeSurv is available at
github.com/ayoung31/DeSurv. Code and processed data used in this study
are available at github.com/ayoung31/DeSurv-paper.

\showmatmethods
\showacknow

\pnasbreak

\section*{Versioning}\label{versioning}
\addcontentsline{toc}{section}{Versioning}

\begin{verbatim}
## DeSurv package version: 1.0.1
## DeSurv git branch: main
## DeSurv git commit: 370c88aa9a89e0c71507fbb161d226c7e2e4c61f
## Paper git branch: main
## Paper git commit: 21b323b4685ef3b4443e34b7e4f106dc7e8096b4
\end{verbatim}

\phantomsection\label{refs}
\begin{CSLReferences}{0}{1}
\bibitem[\citeproctext]{ref-collisson2011subtypes}
\CSLLeftMargin{1. }%
\CSLRightInline{Collisson EA, et al. (2011) Subtypes of pancreatic
ductal adenocarcinoma and their differing responses to therapy.
\emph{Nature medicine} 17(4):500--503.}

\bibitem[\citeproctext]{ref-nguyen2024fourteen}
\CSLLeftMargin{2. }%
\CSLRightInline{Nguyen H, Nguyen H, Tran D, Draghici S, Nguyen T (2024)
Fourteen years of cellular deconvolution: Methodology, applications,
technical evaluation and outstanding challenges. \emph{Nucleic Acids
Research} 52(9):4761--4783.}

\bibitem[\citeproctext]{ref-lee1999learning}
\CSLLeftMargin{3. }%
\CSLRightInline{Lee DD, Seung HS (1999) Learning the parts of objects by
non-negative matrix factorization. \emph{nature} 401(6755):788--791.}

\bibitem[\citeproctext]{ref-moffitt2015virtual}
\CSLLeftMargin{4. }%
\CSLRightInline{Moffitt RA, et al. (2015) Virtual microdissection
identifies distinct tumor-and stroma-specific subtypes of pancreatic
ductal adenocarcinoma. \emph{Nature genetics} 47(10):1168--1178.}

\bibitem[\citeproctext]{ref-peng2019novo}
\CSLLeftMargin{5. }%
\CSLRightInline{Peng XL, Moffitt RA, Torphy RJ, Volmar KE, Yeh JJ (2019)
De novo compartment deconvolution and weight estimation of tumor samples
using DECODER. \emph{Nature communications} 10(1):4729.}

\bibitem[\citeproctext]{ref-brunet2004metagenes}
\CSLLeftMargin{6. }%
\CSLRightInline{Brunet J-P, Tamayo P, Golub TR, Mesirov JP (2004)
Metagenes and molecular pattern discovery using matrix factorization.
\emph{Proceedings of the national academy of sciences}
101(12):4164--4169.}

\bibitem[\citeproctext]{ref-Bailey2016}
\CSLLeftMargin{7. }%
\CSLRightInline{Bailey P, Chang DK, et al. (2016)
\href{https://doi.org/10.1038/nature16965}{Genomic analyses identify
molecular subtypes of pancreatic cancer}. \emph{Nature}
531(7592):47--52.}

\bibitem[\citeproctext]{ref-bair2004semi}
\CSLLeftMargin{8. }%
\CSLRightInline{Bair E, Tibshirani R (2004) Semi-supervised methods to
predict patient survival from gene expression data. \emph{PLoS biology}
2(4):e108.}

\bibitem[\citeproctext]{ref-cook2007fisher}
\CSLLeftMargin{9. }%
\CSLRightInline{Cook RD (2007) Fisher lecture: Dimension reduction in
regression. \emph{Statistical Science} 22(1):1--26.}

\bibitem[\citeproctext]{ref-rashid2020purity}
\CSLLeftMargin{10. }%
\CSLRightInline{Rashid NU, et al. (2020) Purity independent subtyping of
tumors (PurIST), a clinically robust, single-sample classifier for tumor
subtyping in pancreatic cancer. \emph{Clinical Cancer Research}
26(1):82--92.}

\bibitem[\citeproctext]{ref-peng2024determination}
\CSLLeftMargin{11. }%
\CSLRightInline{Peng XL, et al. (2024) Determination of permissive and
restraining cancer-associated fibroblast (DeCAF) subtypes.
\emph{bioRxiv}.}

\bibitem[\citeproctext]{ref-aran2015systematic}
\CSLLeftMargin{12. }%
\CSLRightInline{Aran D, Sirota M, Butte AJ (2015) Systematic pan-cancer
analysis of tumour purity. \emph{Nature Communications} 6:8971.}

\bibitem[\citeproctext]{ref-collisson2019molecular}
\CSLLeftMargin{13. }%
\CSLRightInline{Collisson EA, Bailey P, Chang DK, Biankin AV (2019)
Molecular subtypes of pancreatic cancer. \emph{Nature reviews
Gastroenterology \& hepatology} 16(4):207--220.}

\bibitem[\citeproctext]{ref-Schwarzova2023}
\CSLLeftMargin{14. }%
\CSLRightInline{Schwarzová L, Bouchal P, Brychtová S, Hrstka R (2023)
Stroma-rich bladder cancers: Biological features, clinical relevance,
and therapeutic targeting. \emph{Cancers} 15(5):1503.}

\bibitem[\citeproctext]{ref-bair2006supervised}
\CSLLeftMargin{15. }%
\CSLRightInline{Bair E, Hastie T, Paul D, Tibshirani R (2006) Prediction
by supervised principal components. \emph{Journal of the American
Statistical Association} 101(473):119--137.}

\bibitem[\citeproctext]{ref-arora2020survclust}
\CSLLeftMargin{16. }%
\CSLRightInline{Arora A, Olshen AB, Seshan VE, Shen R (2020)
Surv{C}lust: An integrative survival-weighted clustering method for
multi-omic data. \emph{bioRxiv}.
doi:\href{https://doi.org/10.1101/2020.09.04.283838}{10.1101/2020.09.04.283838}.}

\bibitem[\citeproctext]{ref-gaujoux2010flexible}
\CSLLeftMargin{17. }%
\CSLRightInline{Gaujoux R, Seoighe C (2010) A flexible r package for
nonnegative matrix factorization. \emph{BMC bioinformatics} 11(1):367.}

\bibitem[\citeproctext]{ref-huang2020low}
\CSLLeftMargin{18. }%
\CSLRightInline{Huang Z, Salama P, Shao W, Zhang J, Huang K (2020)
Low-rank reorganization via proportional hazards non-negative matrix
factorization unveils survival associated gene clusters. \emph{arXiv
preprint arXiv:200803776}.}

\bibitem[\citeproctext]{ref-le2025survnmf}
\CSLLeftMargin{19. }%
\CSLRightInline{Le Goff V, et al. (2025) SurvNMF: Non-negative matrix
factorization supervised for survival data analysis. PhD thesis
(Institut Pasteur Paris; CEA).}

\bibitem[\citeproctext]{ref-damrauer2014intrinsic}
\CSLLeftMargin{20. }%
\CSLRightInline{Damrauer JS, et al. (2014) Intrinsic subtypes of
high-grade bladder cancer reflect the hallmarks of breast cancer
biology. \emph{Proceedings of the National Academy of Sciences}
111(8):3110--3115.}

\bibitem[\citeproctext]{ref-Hoadley2018}
\CSLLeftMargin{21. }%
\CSLRightInline{Hoadley KA, Yau C, et al. (2018)
\href{https://doi.org/10.1016/j.cell.2018.03.022}{Cell-of-origin
patterns dominate the molecular classification of 10,000 tumors from 33
types of cancer}. \emph{Cell} 173(2):291--304.}

\bibitem[\citeproctext]{ref-frigyesi2008nmf}
\CSLLeftMargin{22. }%
\CSLRightInline{Frigyesi A, Höglund M (2008) Non-negative matrix
factorization for the analysis of complex gene expression data:
Identification of clinically relevant tumor subtypes. \emph{Cancer
Informatics} 6:275--292.}

\bibitem[\citeproctext]{ref-tomczak2015review}
\CSLLeftMargin{23. }%
\CSLRightInline{Tomczak K, Czerwińska P, Wiznerowicz M (2015) Review the
cancer genome atlas (TCGA): An immeasurable source of knowledge.
\emph{Contemporary Oncology/Wsp{ó}{ł}czesna Onkologia} 2015(1):68--77.}

\bibitem[\citeproctext]{ref-ellis2013clinical}
\CSLLeftMargin{24. }%
\CSLRightInline{Ellis M, et al. (2013) Clinical proteomic tumor analysis
consortium (CPTAC): Connecting genomic alterations to cancer biology
with proteomics: The NCI clinical proteomic tumor analysis consortium.
\emph{Cancer Discov} 3:1108--1112.}

\bibitem[\citeproctext]{ref-Maurer2019}
\CSLLeftMargin{25. }%
\CSLRightInline{Maurer C, et al. (2019)
\href{https://doi.org/10.1136/gutjnl-2018-317706}{Experimental
microdissection enables functional harmonisation of pancreatic cancer
subtypes}. \emph{Gut} 68(6):1034--1043.}

\bibitem[\citeproctext]{ref-tishby1999information}
\CSLLeftMargin{26. }%
\CSLRightInline{Tishby N, Pereira FC, Bialek W (1999) The information
bottleneck method. \emph{Proceedings of the 37th Annual Allerton
Conference on Communication, Control, and Computing}:368--377.}

\end{CSLReferences}



% Bibliography
% \bibliography{pnas-sample}

\end{document}
